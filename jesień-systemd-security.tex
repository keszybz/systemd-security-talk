\documentclass[serif]{beamer}
%\usepackage[utopia]{mathdesign}
%\usepackage[no-math]{fontspec}
%\setmainfont{Liberation Serif}

\usepackage{minted}
% \usepackage{redhat}

\title[Systemd security features]{Securing your daemons using systemd}
\author{Zbigniew Jędrzejewski-Szmek}
\institute{%
  \includegraphics{beamer-themeredhat/redhat.png}\\
  \medskip
  \textit{zbyszek@in.waw.pl}
}
\date{\tiny 10.11.18}

\begin{document}
\begin{frame}
\titlepage % Print the title page as the first slide
\end{frame}

\begin{frame}
  \frametitle{Before we begin...\\Why use systemd for this at all?}
  \begin{itemize}
  \item centralization
  \item abstraction of hardware architecture / kernel version
  \item unprivileged operation
  \end{itemize}
\end{frame}

\begin{frame}[fragile]
  \frametitle{Before we begin...\\Unit files}

  \begin{minted}{ini}
    # /etc/systemd/system/mydaemon.service
    [Service]
    ExecStart=/usr/local/bin/mydaemon
  \end{minted}
  \begin{minted}{console}
    $ systemd-run mydaemon.service
  \end{minted}

  \medskip

  \begin{minted}{console}
    $ systemd-run /usr/local/bin/mydameon
  \end{minted}
\end{frame}

%% \begin{frame}
%%   \frametitle{Before we begin...\\\texttt{systemd-run}}

%% \end{frame}

\begin{frame}
  \frametitle{Basics}
  \texttt{User=}
\end{frame}

\begin{frame}
  \frametitle{Limiting access to the file system}
  \begin{itemize}
  \item \texttt{ProtectHome=yes|read-only}
  \item \texttt{ProtectSystem=yes|full|strict}

  \item \texttt{InaccessiblePaths=}
  \item \texttt{ReadOnlyPaths=}
  \item \texttt{ReadWritePaths=}
  \end{itemize}
\end{frame}

\begin{frame}
  \frametitle{Limiting access to the file system\\a better way}
  \begin{itemize}
  \item \texttt{RuntimeDirectory=foo}            \# /run/foo/
  \item \texttt{StateDirectory=foo}              \# /var/lib/foo/
  \item \texttt{CacheDirectory=foo}              \# /var/cache/foo/
  \item \texttt{LogsDirectory=foo}               \# /var/log/foo/
  \item \texttt{ConfigurationDirectory=foo}      \# /etc/foo/
  \end{itemize}
\end{frame}

\begin{frame}[fragile]
  \begin{minted}{console}
    $ sudo systemd-run -t -p User=zbyszek -p RuntimeDirectory=foo ls -ld /run/foo
  \end{minted}

  \begin{itemize}
  \item automatic \textit{creation} and \textit{ownership}
  \item automatic \textit{removal}
  \end{itemize}
\end{frame}

\begin{frame}
  \begin{itemize}
  \item \texttt{PrivateTmp=yes}
  \end{itemize}
\end{frame}

\begin{frame}[fragile]
  \frametitle{User creation on demand?}
  \begin{itemize}
  \item \texttt{DynamicUser=yes}
  \end{itemize}

  \begin{minted}{console}
$ echo -e 'asdf\n\asdf' | \
  systemd-run --pipe -p DynamicUser=1 \
    bash -c 'whoami; grep .' | \
  systemd-run --pipe -p DynamicUser=1 \
    bash -c 'whoami; grep .' | \
  systemd-run --pipe -p DynamicUser=1 \
    bash -c 'whoami; grep .'
  \end{minted}
\end{frame}

\begin{frame}
  \begin{itemize}
  \item \texttt{PrivateNetwork=yes}
  \end{itemize}

  ``\texttt{PrivateNetwork=yes} is the recommeded way to run network servers''
\end{frame}

\begin{frame}[c]
  \frametitle{Socket Activation}

  \centering
  Two types of socket activation:

  \medskip
  
  \begin{columns}
    \begin{column}{0.5\textwidth}
      \texttt{Accept=yes}\\
      → a single instance of the service is started for each connection\\
      → ``wait'' under inetd/xinetd
    \end{column}
    \begin{column}{0.5\textwidth}
      \texttt{Accept=no}\\
      → a single instance of the service is started for each connection\\
      → ``nowait'' under inetd/xinetd
    \end{column}
  \end{columns}

\end{frame}

\begin{frame}
  \begin{itemize}
  \item \texttt{PrivateDevices=yes}
  \item \texttt{RestrictAddressFamilies=}
  \item \texttt{NoNewPrivileges=}
  \item \texttt{ProtectKernelTunables=}
  \item \texttt{SystemCallArchitectures=native|x86\_64|i386|...}
  \item \texttt{LockPersonality=yes}
  \item \texttt{MemoryDenyWriteExecute=yes}
  \end{itemize}
\end{frame}

\begin{frame}
  \frametitle{System call filtering}
  ``seccomp mode 2''
  \begin{itemize}
  \item \texttt{libseccomp}
  \item \texttt{syscall1 | syscall2 | @group}
  \item \texttt{@basic-io}
  \item \texttt{@obsolete}
  \end{itemize}
\end{frame}

\begin{frame}
  \frametitle{Per-service network firewall}
  
  \begin{itemize}
  \item \texttt{IPAddressAllow=}
  \item \texttt{IPAddressAllow=}
  \end{itemize}
\end{frame}

\begin{frame}[fragile]
  \frametitle{Upcoming features}

  \begin{minted}{console}
$ systemd-analyze-security systemd-resolved.service
  \end{minted}
\end{frame}

\begin{frame}
  \frametitle{Upcoming features, ctd}

  \begin{itemize}
  \item \texttt{IPAddressAllow=}
  \end{itemize}
\end{frame}

\end{document}
